\subsection{機能概要}
このAndoroidアプリでは以下の機能を導入します。\\ \\
(1)当日の授業情報\\
\hspace{5mm}
\begin{minipage}[h]{145mm}
 当日の時間割がトップページに表示され、確認できます。
\end{minipage}
\\ \\
(2)履修登録情報\\
\hspace{5mm}
\begin{minipage}[h]{145mm}
 Web上で登録しておいた履修情報の一覧が確認できます。
\end{minipage}
\\ \\
(3)成績\\
\hspace{5mm}
\begin{minipage}[h]{145mm}
 自分の成績を確認できます。成績の表示は「科目区分別単位修得状況」「年度・学期別単位修得状況」「期間GPA」の3つの表示方法を選択できます。
\end{minipage}
\\ \\
(4)お知らせ\\
\hspace{5mm}
\begin{minipage}[h]{145mm}
 学校からのお知らせを閲覧できます。お知らせごとに項目分けしています。
\end{minipage}
\\ \\
(5)リンク\\
\hspace{5mm}
\begin{minipage}[h]{145mm}
 学校の関連ホームページをブラウザで開くことができます。リンクは「シラバス」「KUTLMS」「WebMailService」の3つです。
\end{minipage}
\\ \\
(6)テストカウントダウン\\
\hspace{5mm}
\begin{minipage}[h]{145mm}
 手動で設定しておくと,テストまであと何日かを表示してくれます。
\end{minipage}
\\ \\
(7)設定\\
\hspace{5mm}
\begin{minipage}[h]{145mm}
  このアプリのテストカウントダウン機能の設定と画面の見た目の設定ができます。
\end{minipage}

\subsection{前提条件}
このシステムは以下を前提条件とします。
\begin{itemize}
\item 4クオータ制を導入している学校を対象とする
\item 作成したアプリをシステムを利用している学生にダウンロードしてもらう
\end{itemize}
