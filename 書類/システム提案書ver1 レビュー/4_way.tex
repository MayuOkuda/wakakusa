前項で説明した事柄に関して、具体的な方法を以下に示します。
%\begin{itemize}
%\item 既存学生支援システムに対応したAndroidアプリケーション
%\item Androidアプリケーションとサーバの連携システム
%\end{itemize}
\begin{itemize}
\item 簡易な情報取得\\
 個人情報の保護のための認証機能は、指紋認証など端末に付随されている認証技術を利用します。アプリにおいて、一度ログインされたアプリはログアウトされるまで、その人物に関する情報を保持します。これにより、セキュリティ上、毎回行われるIDやパスワード認証を実施せず、素早い情報取得を実現します。
  %携帯の指紋認証とかされてるから、ある程度のセキュリティは確保されているよねってくだりをどこかに入れたかった。
\item 他Webサイトへの遷移\\
 これは、授業内容や書類提出先など学生に関連した他Webサイトに対して、Webブラウザ上の該当ページを表示するリンクなどを付属し、それらの情報取得を容易にするための項目を提供します。
\item ネットワーク環境に依存しない同情報取得\\
 このアプリは、以前にアクセスした際のデータを端末側に保持し、不安定なネットワーク状況下においても、前回アクセスした内容と同等な情報取得を可能にします。
\end{itemize}
