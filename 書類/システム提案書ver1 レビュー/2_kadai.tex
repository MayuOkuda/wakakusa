%アンケートを行った結果、現在のシステムに不満を持っている人がいることが分かりました。既存のシステムではオンライン状況でしか既存のシステムにある情報を確認できません。その為、オフライン上で更新された履修状況を確認できる様に、アプリ内での情報保持を行う事でネットワーク経由でのアクセスを行う必要が無いと言ったメリットがあります。\\
%また、履修状況のスケジュール管理を行える様になる事で、授業への遅刻・欠席が減り、学生の授業参加率を向上させる事が可能になります。
%(別府)具体的な数値がなく、グラフを貼るわけでもないのにアンケートのことをのべるのはどうかと、あと解決策をのべるところなのか微妙。
学生は、自身の履修状況や学校からの連絡を確認し、それに合わせた生活を送らなければなりません。しかし、高知工科大学生を対象に実施したアンケート結果では、既存のシステムに対して不満を持っている人が半数以上確認され、より便利な機能を欲する声も多く挙げられました。\\
 また、Web上のシステムを利用するには、オンライン状況でアクセスを行うという前提が存在します。そのため、回線速度が遅い、ネットワークが混み合っているなどといった不安定な通信状況下では、自身のスケジュールやお知らせを確認することが困難です。この課題の要因には、「月ごとのデータ使用量の超過」や「同時刻におけるシステムへのアクセス数の多さ」などが挙げられます。\\
 前者は学生が自身の携帯端末を使用した時に限られますが、通常、多様な情報を取得するために使用している人々も多く存在するため、システムの円滑な利用にあたって、携帯端末からのアクセスを考慮に入れる必要があります。後者に関しては、ある程度の時間経過を待つことによって解消される可能性もありますが、緊急の場合に対応できないという問題点があります。\\
 したがって、従来、学生に提供されているシステムだけでは、学生が自身のスケジュール管理を行い、より快適な学生生活を送るための支援として不十分であると考えられます。
