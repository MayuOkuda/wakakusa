システムにかかる費用を以下の表1に示します。本システムで使用するサーバは学校のシステムに使用されているサーバを用いることを想定しているため、サーバの購入費は考慮しないものとしています。

\begin{table}[htb]
  \begin{center}
    \caption{システム化にかかる費用}
    \begin{tabular}{|c|r|c|r|c|} \hline
      項目 & 単価(円) & 数量 & 金額(円)& 備考 \\ \hline \hline
      開発人件費 & 5,000 & 720人日 & 3,600,000 & 8人×90日 \\ \hline
      導入費 & 5,000 & 56人日 & 280,000 & 8人×7日 \\ \hline
      運用・保守費 & 40,000 & 12か月 & 480,000 & 月一回のメンテナンス \\ \hline
      合計 & & & 4,360,000 & \\\hline
    \end{tabular}
  \end{center}
\end{table}


システム提案による効果の利益を以下に示します。前提条件として、利用者が本アプリケーションを使用できる端末を所持していることを想定します。この場合、簡易な情報取得によるスケジュール管理が可能なため、授業に対する遅刻・欠席回数の減少、学生の単位取得の向上が見込まれます。また、サーバ側にてメンテナンスが行われている場合でも、アプリを用いることで、事前に取得した履修状況などの情報を確認する事が可能になります。加えて、ネットワーク通信が不安定な場合でも、安定して閲覧できる事で利用者の心的ストレスをカットする事ができると考えられます。
さらに、安定した情報取得が可能なため、履修情報などの印刷を行う必要性が減り、その際にかかる資源の節約も可能であると考えられます。
